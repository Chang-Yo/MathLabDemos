%!TEX program = xelatex

% 告诉LaTeX我们正在使用的是一个中文文章类,并设置了小四字号
\documentclass[zihao=-4]{ctexart} 

% --- 导言区 (设置格式和导入宏包) ---

% 1. 字体设置 (使用 XeLaTeX 编译器)
% 设定中文、英文默认字体,满足你的要求
\usepackage{fontspec}
\setmainfont{Times New Roman} % 英文默认字体
\setCJKmainfont{SimSun}        % 中文默认字体 (宋体)

% 2. 页面边距设置 (上 2.5, 下 2.5, 左 2.8, 右 2.2 cm)
\usepackage[
    top=2.5cm,
    bottom=2.5cm,
    left=2.8cm,
    right=2.2cm
]{geometry}

% 3. 行距设置 (1.5 倍行距)
\usepackage{setspace}
\onehalfspacing 

% 4. 段落格式设置 (首行缩进 2 字符, 段前段后 0 行)
\usepackage{parskip} % 避免段落间距干扰
\setlength{\parindent}{2em} % 设置首行缩进 2 个中文字符 (2em)
% CTeX 文档类默认段前段后是 0 行,这里通常不需要额外设置。

% 5. 论文题目格式 (小三, 黑体, 加粗, 居中, 18 磅)
\usepackage{titlesec}
\titleformat{\section}[block]{\centering\bfseries\zihao{4}\heiti}{}{0em}{} % 设置一级标题格式(用作一级小标题)

% 确保小标题使用黑体加粗小四(zihao=4)
\titleformat{\section}{\zihao{4}\bfseries\heiti}{\thesection}{1em}{} % 一级小标题格式
\titleformat{\subsection}{\zihao{4}\bfseries}{\thesubsection}{1em}{} % 二级小标题格式(如不需要可省略)


% 定义题目、作者等元信息
\title{
    % 题目字体设置:18pt 小三(zihao=-3), 黑体, 加粗
    \fontsize{18pt}{24pt}\selectfont\heiti\textbf{你的大作业的题目放在这里}
}
\author{你的姓名 / 你的学号}
\date{\today} % 默认显示当前日期,或使用 \date{} 清空不显示

% --- 正文区 ---
\begin{document}

% 插入标题
\maketitle

% 正文从这里开始:小四 (zihao=-4), 宋体 (默认)
% 记得要使用 XeLaTeX 进行编译!

\section{一级小标题放在这里} % 这是你的“一级小标题”:小四 黑体 加粗
\label{sec:introduction}

这是正文的第一段。正文要求是小四(12磅)宋体,行距1.5倍,首行缩进2字符。在\texttt{ctexart}中,默认字号为小四(zihao=-4)。本模板已经通过\texttt{setCJKmainfont\{SimSun\}}将中文设置为宋体。你可以开始在这里输入内容。

这是正文的第二段,你不需要手动按空格来实现缩进,LaTeX会根据\texttt{\string\setlength\{\string\parindent\}\{2em\}}自动为你处理好。这个结构化的排版系统将为你节省大量时间。

\subsection{二级小标题 (自动编号)}
\label{sec:methodology}
\subsubsection{三级小标题}

你的正文内容。你可以继续输入更多内容。

\section{另一个一级小标题}
\label{sec:conclusion}

你的结论部分。

\end{document}